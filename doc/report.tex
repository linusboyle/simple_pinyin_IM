\documentclass{article}
\usepackage{CJK}
\usepackage{amsmath}
\usepackage[tikz]{mdframed}

%%%%%%%%%% Start TeXmacs macros
\catcode`\<=\active \def<{
\fontencoding{T1}\selectfont\symbol{60}\fontencoding{\encodingdefault}}
\catcode`\>=\active \def>{
\fontencoding{T1}\selectfont\symbol{62}\fontencoding{\encodingdefault}}
\newcommand{\tmnote}[1]{\thanks{\textit{Note:} #1}}
\newcommand{\tmop}[1]{\ensuremath{\operatorname{#1}}}
\newcommand{\tmstrong}[1]{\textbf{#1}}
\newcommand{\tmsubtitle}[1]{\thanks{\textit{Subtitle:} #1}}
\newenvironment{itemizedot}{\begin{itemize} \renewcommand{\labelitemi}{$\bullet$}\renewcommand{\labelitemii}{$\bullet$}\renewcommand{\labelitemiii}{$\bullet$}\renewcommand{\labelitemiv}{$\bullet$}}{\end{itemize}}
\newenvironment{tmindent}{\begin{tmparmod}{1.5em}{0pt}{0pt}}{\end{tmparmod}}
\newenvironment{tmparmod}[3]{\begin{list}{}{\setlength{\topsep}{0pt}\setlength{\leftmargin}{#1}\setlength{\rightmargin}{#2}\setlength{\parindent}{#3}\setlength{\listparindent}{\parindent}\setlength{\itemindent}{\parindent}\setlength{\parsep}{\parskip}} \item[]}{\end{list}}
\mdfsetup{linecolor=black,linewidth=0.5pt,skipabove=0.5em,skipbelow=0.5em,hidealllines=true,
innerleftmargin=0pt,innerrightmargin=0pt,innertopmargin=0pt,innerbottommargin=0pt}
\newmdenv[topline=true,bottomline=true,innertopmargin=1ex,innerbottommargin=1ex]{tmbothlined}
%%%%%%%%%% End TeXmacs macros

\begin{document}
\begin{CJK*}{UTF8}{gbsn}

\title{
  人工智能导论 作业报告
  \tmsubtitle{拼音输入法}
  \tmnote{2019年3月30日}
}

\author{
  韩志磊
  \tmnote{计76,学号2017011442}
}

\maketitle

\section{原理与分析}

\subsection{算法}

\subsubsection{统计算法}

{\leftaligned{\begin{tmindent}
  \
  
  为实现拼音序列到汉字的转换,基于预先给定的语料,可以采用概率统计的方法。{\tmstrong{假定样本具有足够的代表性}},那么,通过统计样本中各字的出现频率,就可以知晓其大致的实际使用频率。然而,这样只能保证单字的出现概率最大,没有考虑上下文的影响。如果我们不对汉语的语法和语义进行分析(现有的语料也没这个条件),单纯就文本表观的频度而言,就需要一些更为复杂的概率计算。
  
  假定汉字的出现概率可以用古典概型计算,事件$O_n$代表输入长度为n的拼音串$\alpha_{1,
  2, 3 \ldots
  n}$,事件$\omega_i$代表第i位拼音代表的字符,$\omega$代表总的长度为n的字符串。我们希望概率$P
  (\omega_{} |O_n)$最小,根据贝叶斯公式:
  \[ P (\omega |O_n) = \frac{P (O_n | \omega) \times P (\omega)}{P (O_n)} \]
  
  
  在一次试验中,$P (O_n)$是定值;而$P (O_n |
  \omega)$是「某个字符串读音为O的概率」,{\tmstrong{在不考虑多音字的情况下}},这个概率是1。因此,我们只需要求符合给定读音的某个字符串,它本身的出现概率最大。
  
  由乘法公式立即可得:
  \[ P (\omega) = P (\omega_1 \omega_2 \omega_3 \cdots \omega_n) = \sum^n_{i =
     1} P (\omega_i | \omega_1 \omega_2 \cdots \omega_{i - 1}) \]
  
  
  精确地计算需要每个i元词组出现的频率,这意味着指数级的预处理复杂度(或运行期指数级时间复杂度),对于这个问题是不可接受的。考虑到大部分时候,没有必要考虑已出现的所有前缀对当前字符判定的影响,可以适当对条件进行弱化。前面提到的只考虑单字概率的模型,就是不考虑条件项的结果。由此,我们可以从二元模型开始,即只考虑当前字符的前一字符的影响。
  \[ P (\omega) = \sum_{i = 1}^n P (\omega_i | \omega_{i - 1}) = \sum_{i =
     1}^n \frac{P (\omega_{\iota} \omega_{i - 1})}{P (\omega_{i - 1})} =
     \sum^n_{i = 1} \frac{\# \omega_i \omega_{i - 1}}{\# \omega_{i - 1}} \]
  
  
  如此一来,我们只需要统计大约6000*6000个词组的出现频率即可。
  
  \ 
\end{tmindent}}}

\subsubsection{识别算法}

\begin{tmindent}
  \
  
  现在我们已经完成了对原始语料的统计,并得到了一个二维矩阵occurrence用以标识二元词组的出现次数、一个一位数组count记录单个字符的出现次数,并且知道语料库的总大小。输入一个拼音的集合,如何给出概率最大的字符串呢?
  
  在原型程序中,我使用的算法可以大致描述如下:
  
  算法逐拼音进行处理,令$W_i$为第i个拼音对应的可能字符集合,$P
  (\omega)$为$\omega$对应的概率。则对于每一个$\omega \in
  W_i$,考虑每一个$x \in W_{i - 1}$,有
  \[ P (\omega) = \max \left( P (x) \cdot \frac{\tmop{occurrence} [x]
     [\omega]}{\tmop{count} [x]} \right), \forall x \in W_{i - 1} \wedge
     \tmop{occurrence} [x] \neq 0 \]
  
  
  并且假定
  \[ P (\omega) = \frac{\tmop{count} [\omega]}{\tmop{TotalNumber}} \forall
     \omega \in W_1 \]
  
  
  算法从$W_1$开始,至$W_n$结束。最终的结果就是$\max (P
  (W_n))$对应的字符串。要回溯这个字符串并不困难,只要在算法进行时用指针记录使得当前P值最大的前一字符即可。(详见代码)
  
  该过程的伪代码如下:
  
  {\rendercode{PROCEDURE FIND\_BEST(PINYINS)
  
  BEGIN
  
  \ \ \ PTABLE <- []
  
  \ \ \
  
  \ \ \ CHARS <- LETTERSTABLE[PINYINS[0]]
  
  \ \ \ FTABLE <- []
  
  \ \ \ FOR CHAR IN CHARS
  
  \ \ \ DO
  
  \ \ \ \ \ \ \ FTABLE.ADD(<COUNT[CHAR]/TOTALNUMBER,NULL,CHAR>)
  
  \ \ \ DONE
  
  \ \ \ PTABLE.ADD(FTABLE)
  
  \
  
  \ \ \ FOR PINYIN IN PINYINS[1,2,3...]
  
  \ \ \ DO
  
  \ \ \ \ \ \ \ LTABLE <- []
  
  \ \ \ \ \ \ \ FOR LCHARS IN LETTERSTABLE[PINYIN]
  
  \ \ \ \ \ \ \ DO
  
  \ \ \ \ \ \ \ \ \ \ \ L\_PROBABILITY <- 0
  
  \ \ \ \ \ \ \ \ \ \ \ PREV\_MAX <- NULL
  
  \ \ \ \ \ \ \ \ \ \ \ PREV\_TABLE = PTABLE.BACK()
  
  \
  
  \ \ \ \ \ \ \ \ \ \ \ FOR PREV IN PREV\_TABLE
  
  \ \ \ \ \ \ \ \ \ \ \ DO
  
  \ \ \ \ \ \ \ \ \ \ \ \ \ \ \ N\_PROBABILITY <- PREV.FIRST() *
  OCCURRENCE[PREV][LCHARS] / COUNT[PREV]
  
  \ \ \ \ \ \ \ \ \ \ \ \ \ \ \ IF N\_PROBABILITY > L\_PROBABILITY
  
  \ \ \ \ \ \ \ \ \ \ \ \ \ \ \ THEN
  
  \ \ \ \ \ \ \ \ \ \ \ \ \ \ \ \ \ \ \ L\_PROBABILITY <- N\_PROBABILITY
  
  \ \ \ \ \ \ \ \ \ \ \ \ \ \ \ \ \ \ \ PREV\_MAX <- PREV
  
  \ \ \ \ \ \ \ \ \ \ \ \ \ \ \ DONE
  
  \ \ \ \ \ \ \ \ \ \ \ DONE
  
  \ \ \ \ \ \ \ \ \ \ \ LTABLE.ADD<L\_PROBABILITY,PREV\_MAX,LCHARS>
  
  \ \ \ \ \ \ \ DONE
  
  \ \ \ \ \ \ \ PTABLE.ADD<LTABLE>
  
  \ \ \ DONE
  
  END}}
\end{tmindent}

\subsection{实验结果}

\subsubsection{案例}

\begin{tmindent}
  \
  
  在实现了上述算法(代码说明见后)之后,得到了较为准确、迅速的原型程序。试选取其中成功的例子为例:
  
  \
  
  \begin{tmbothlined}
    \item 清华大学计算机系
    
    \item 北京市人民政府
    
    \item 中国特色社会主义
    
    \item 爱国主义统一战线
    
    \item 五十六个民族
    
    \item 我还在上班呢
    
    \item 你今晚有空吗
    
    \item 精准扶贫
    
    \item 反法西斯战争
    
    \item 我的手机没电了
    
    \item 菜市场里人很多
  \end{tmbothlined}
  
  \
  
  \
  
  {\tmstrong{不成功}}例子譬如:
  
  \begin{tmbothlined}
    \begin{itemizedot}
      \item 义勇军进行曲
      
      >>义勇军进行区
      
      \item 人民当家做主
      
      >>人民党家做主
      
      \item 你能借我你的铅笔吗
      
      >>你能解我你的铅笔马
      
      \item 大家有什么想说的吗
      
      >>大家有什么想说的马
      
      \item 火箭成功发射上天
      
      >>和建成功发射上天
      
      \item 和稀泥
      
      >>获悉尼
      
      \item 今天回家比较晚
      
      >>今天回家比较完
    \end{itemizedot}
  \end{tmbothlined}
\end{tmindent}


\subsubsection{分析}

\begin{tmindent}
  \
  
  由上可以看出,由于语料库的限制,原型程序所能准确识别只有某一类别的词组。譬如,由于给定的语料库绝大部分是新闻,因此时政、经济等方面的词组容易被识别;而相较之下,日常生活用语则较难被识别。当然,目前的算法也可以一部分地选择出有多个词组的最佳的字符串,像「菜市场里人很多」这样的句子就可以被识别,说明存在着概率控制的简单分词功能。
  
  但这毕竟不是真正的分词,「人民当家做主」这种词,原本可以比较简单地分出三个词组,但由于概率模型中「国民党」这个词出现的频率太大,占据了主要的乘积项,导致识别失败。「义勇军进行曲」也是同理。
  
  由此可以看出基于概率模型的输入法本身就有许多问题。一者受语料限制太大,二者受个别辞藻影响过深。此外还有缺失语义分析的问题:上面的例子中,「吗」这个语气词并不能准确地选择,原因在于对于「问句」这种形式无法判定。而且一旦输入的句子里语料的偏差越大、长度越长,语义上的问题就越加凸显。
  
  汉语的规则比较复杂,很难用上下文无关的语法来限定候选词的范围。不过,拥有语法分析的模型无疑能更好地处理相关问题(虽然有些时候会是事倍功半,因为汉语的反常现象很多)。
  
  多音字是另一个非常恼人的问题。「火箭」这种词的出现频率没有「和建」高,因此识别失败;但是「和」读huó的概率极小,显然此处是不合理的。而「和稀泥」这类的词,由于缺乏字到多音的对应关系,也没办法识别。因此对于实际的输入法程序而言,某种双向的映射关系及数量关系是必须的。
\end{tmindent}

\section{实现}

\subsection{项目说明}

\begin{tmindent}
  \
  
  请注意以下所有文件均为utf-8编码,请不要使用gbk编码运行。
  
  按照说明的要求,最终的结构如下:
  
  \begin{quotation}
    .
    
    ├── bin
    
    │ ├── main
    
    │ └── wrapper.py
    
    ├── data
    
    │ ├── input.txt
    
    │ └── output.txt
    
    ├── src
    
    │ ├── count.cpp
    
    │ ├── data.cpp
    
    │ ├── main.cpp
    
    │ ├── support
    
    │ │ ├── cat.js
    
    │ │ ├── count.js
    
    │ │ ├── letters.txt
    
    │ │ ├── p2l.txt
    
    │ │ ├── pinyin2letter.txt
    
    │ │ ├── process.js
    
    │ │ ├── statistics.js
    
    │ │ ├── translation.js
    
    │ │ └── tr.sh
    
    │ ├── table.cpp
    
    │ └── ver2.cpp
    
    └── todo.txt
  \end{quotation}
  
  说明中的「指定输入、输出文件」的功能只能由脚本wrapper.py实现,主程序main只进行计算。确保操作系统为64位Unix,且有libstdc++库,则可以直接使用编译好的程序。要指定输入、输出文件,向封装脚本传入参数即可:
  \begin{tmindent}
    ./wrapper.py myinut myoutput
  \end{tmindent}
  请确保输入文件是utf8编码、Unix换行符风格。
\end{tmindent}

\subsection{编译说明}

\begin{tmindent}
  \
  
  核心程序以C++编写,预处理与支持脚本以JavaScript编写,包装脚本以Python编写。要完整地完成从语料到可执行程序的过程,请确保安装C++11标准及以上的编译器、Node.js解释器,并预留2GB空间、至少4GB内存。
  
  首先将所有的语料转换到utf-8,复制到support文件夹下,运行cat.js,会生成whole.txt。然后,依次运行process.js、statictics.js、count.js。{\tmstrong{该步大概需要1小时时间完成。}}
  
  将生成的occurrence.txt和count.txt复制到src/源码目录,然后用C++11标准编译该目录下{\tmstrong{除ver2.cpp外}}的所有C++源代码,并链接为可执行文件。{\tmstrong{注意:编译需要大概10分钟,4\~{}7GB内存。建议先生成对象文件,再进行链接。}}
  
  将可执行文件复制到与Python代码同级的目录,以Python2运行脚本即可。
\end{tmindent}

\subsection{优化}

\subsubsection{朴素的优化法}

\begin{tmindent}
  \
  
  main.cpp是经典的实现,同时ver2.cpp也提供了「改进版」。但值得说明的是,改进是相对于某种输入而言的,对于其他的输入,这个改进版反而会使得识别率下降。考虑到鲁棒性的问题,我没有提供ver2的可执行版本。
  
  ver2所做的优化是很朴素的。在不涉及语法、语义、增加语料、导入预设字典、使用第三方库的情况下,我们能利用的,只有occurance和count两个数据而已。从某种意义上来说,如果只有这些信息,所做出的拼音输入法的准确率总是有上限的,更遑论人的主观思想的影响了。ver2思想是模拟「分词」的效果,通过人为地断开低概率词语间的联系,达到某种程度上的词模型。
  
  ver2是这样实现这个不怎么出色的效果的:如果对于第i个拼音而言,无论选择哪一个汉字,其概率都与上一个字的概率的商小于某一个阈值,则从当前例程递归,将下一个字符视为新的句子开头。这样做的目的也很简单:当两个不怎么常用的词组组合时,不会相互影响。
  
  ver2的效果并不好,对于有些字符串,它的确有比较好的效果;不过对于很大部分的普通特有词语,反而会错误地将其断开。这当然是因为这毕竟不是专业的分词器,其内部的原理是概率而不是查表;但是还有一个很重要的原因在于阈值的选取并不简单。要区分「人民代表大会」和「香辣冰淇淋」这两种词,所需要的阈值非常微妙,因为概率之间的差异不大。
  
  总的来说这是比较失败的优化尝试,但在不借助外力的情况下,固定语料的概率模型的最大限度大抵便是如此了。
\end{tmindent}

\subsubsection{可能的优化法}

\begin{tmindent}
  \
  
  优化的方法是很多的。出于时间、精力上的考虑,我只用提供的语料库本身完成了项目,但实际上,这个语料库的偏向性的确非常强,缺少一些流行文化,而且过于正式了。耗费一些寻找和重编译的时间找一个更好的语料库,其代价绝对是值得的。
  
  三元模型可能并没有看上去那么优秀,可以预见采用三元模型的情况下,其准确性也许在某些时候比现在的二元模型还要差一些。因为二元词组的出现频率,在经验中比三元词组多得多。采用三元词组来进行判定,不仅复杂度更高,往往还会「拆散」原本相连的词语。
  
  二元与三元结合的模型也许会更好,即在判定时有限二元,然后以某个衰减系数乘以三元模型的判定结果作为参考。对于更高维度的模型也是一样的。
  
  分词是很不错的办法(或者找一本字典之类的),但是记忆中比较慢。如果不在运行期进行分词,就必须需要很大的存储空间。同样的,也许更好的办法是词、字模型结合。先对语料「存在」的词语进行搜索,如果没有,才进行基于字的判定。{\setthispageheader{{\unchanged}人工智能导论}}
\end{tmindent}

\end{CJK*}
\end{document}
